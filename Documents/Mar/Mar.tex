\documentclass[lang=cn]{elegantpaper}

\title{三月份过程性研究记录}
\author{陶理}
\usepackage{listings}
\begin{document}
\date{March 28, 2024}
\maketitle
\begin{abstract}
该份过程性研究记录(截止至撰写日期)主要记录了作者在寒假以及三月份所完成的研究成果。主要有以下几项:
\begin{enumerate}
    \item 原定研究计划:
    \begin{enumerate}
        \item 对于方言数据进行收集
        \item 对于收集到的方言数据提取
        \item 采用可视化的方式从多维度考量所获得的mfcc的特征
        \item 采用机器学习算法小样本量地训练(但是效果不好)
    \end{enumerate}
    \item 新思路:
    \begin{enumerate}
        \item 可以采用拓扑学,将所获得的数据映射到一个几何图案(大概率多维),然后再通过不同图形之间的映射转化解决问题(还在设想中,具体可见3Blue1Brown的视频以及四月份的过程性研究记录,届时应该能解释清楚并探讨可行性)
    \end{enumerate}
\end{enumerate}
\end{abstract}
\section{数据程序成果(部分)}
\begin{lstlisting}[language=Python]
import os
import soundfile as sf
import librosa
import numpy as np
import matplotlib.pyplot as plt
input_folder = r'D:\TaoLi\Projects\DialectTranslation\Data\mandarin\audio\GeneratedByAI'
output_folder = r'D:\TaoLi\Projects\DialectTranslation\Data\mandarin\MFCC'
for i in range(100):
    file_name = f"mandarin_{i}_AI.wav"
    input_file_path = os.path.join(input_folder, file_name)         
    data, sample_rate = librosa.load(input_file_path)
    mfccs = librosa.feature.mfcc(y=data, sr=sample_rate, n_mfcc=13)
    output_file_path = os.path.join(output_folder, f"mfcc_{i}.npy")
    # np.save(output_file_path, mfccs)
    print(f"MFCC for {file_name} saved to {output_file_path}")
    plt.figure(figsize=(10, 4))
    librosa.display.specshow(mfccs, x_axis='time')
    plt.colorbar()
    plt.title(f'MFCC for {file_name}')
    plt.tight_layout()
    output_image_path = os.path.join(output_folder, f"mfcc_{i}.png")
    plt.savefig(output_image_path)
    plt.close()
    print(f"Visualization for {file_name} saved to {output_image_path}")
\end{lstlisting}

\end{document}