\documentclass{beamer}
\usepackage{ctex}
\usepackage[english]{babel}
\usepackage{natbib}
\usefonttheme[onlymath]{serif}
\bibliographystyle{plain}
\usetheme{CambridgeUS} 
\usecolortheme{seahorse}
\title{低成本普惠型智能家居方言指令转换器的研究}
\author{陶理}
\date{\today}
\begin{document}
\maketitle
\section{引言}
\subsection{研究背景}
\begin{frame}{研究背景}
    \begin{itemize}
        \item 随着普通话的推广,方言的使用范围正在日益减少,方言处于濒临消失的边缘。
        \item 语音识别领域着重于普通话的识别,而对于方言,特别是小众方言的识别受到了忽略。
        \item 使用方言进行智能家居控制的精度受限。
    \end{itemize}
\end{frame}
\subsection{必要性分析}
\begin{frame}{必要性分析}
\end{frame}
\subsection{研究目的}
\subsection{可行性分析}
\begin{frame}{可行性分析}
\end{frame}
\subsection{国内外研究} 
\begin{frame}{国内外研究}
    \begin{itemize}
        \item 吴永焕\cite{WuYonghuan}
    \end{itemize}
\end{frame}
\section{研究方法及过程}
\subsection{研究方法}
\begin{frame}{研究方法}
    \begin{enumerate}
        \item MFCCs (梅尔频率倒谱系数)
        \item SVD (奇异值分解)
        \item OLS (最小二乘法)
        \item 回归算法
    \end{enumerate}
\end{frame}
\subsection{梅尔频率倒谱系数}
\begin{frame}{梅尔频率倒谱系数}
    Mel刻度,反映人耳对频率的感知:
    \begin{equation}
        Mel(f) = 2595 \log_{10}(1+f/700)
    \end{equation}
    加窗,目的为消除边界干扰:
    \begin{equation}
        w(n) = H(n) = 0.54 - 0.46 \cos \left( \frac{2\pi n}{N-1} \right)
    \end{equation}
\end{frame}
\begin{frame}{梅尔频率倒谱系数}
    短时傅里叶变换
    \begin{equation}
        X(m, \omega) = \sum_{n=-\infty}^{\infty} x(n) \cdot w(n-m) \cdot e^{-j\omega n}
    \end{equation}
\end{frame}
\subsection{研究过程}
\begin{frame}{研究过程}
\end{frame}
\section{结果与讨论}
\subsection{结果}
\begin{frame}{结果}
\end{frame}
\subsection{讨论}
\begin{frame}{讨论}
\end{frame}
\section{结论}
\subsection{结论}
\begin{frame}{结论}
\end{frame}
\section{参考文献}
\subsection{参考文献}
\begin{frame}{参考文献}
    \bibliography{D:\\TaoLi\\Projects\\DialectTranslation\\Beamer\\references_beamer}
\end{frame}
\end{document}